\usepackage[english,ngerman,german]{babel}

\makeatletter
\newcommand{\newlanguagecommand}[1]{%
  \newcommand#1{%
    \@ifundefined{\string#1\languagename}
      {``No def of \texttt{\string#1} for \languagename''}
      {\@nameuse{\string#1\languagename}}%
  }%
}
\newcommand{\addtolanguagecommand}[3]{%
  \@namedef{\string#1#2}{#3}}
\makeatother

\newlanguagecommand{\algo}
\addtolanguagecommand{\algo}{english}{Algorithm}
\addtolanguagecommand{\algo}{ngerman}{Algorithmus}
\newlanguagecommand{\loa}
\addtolanguagecommand{\loa}{english}{List of Algorithms}
\addtolanguagecommand{\loa}{ngerman}{Algorithmen}
\newlanguagecommand{\abbr}
\addtolanguagecommand{\abbr}{english}{List of Abbreviations}
\addtolanguagecommand{\abbr}{ngerman}{Abk\"urzungsverzeichnis}
\newlanguagecommand{\uni}
\addtolanguagecommand{\uni}{english}{University of Bamberg}
\addtolanguagecommand{\uni}{ngerman}{Otto-Friedrich-Universit\"at Bamberg}
\newlanguagecommand{\chair}
\addtolanguagecommand{\chair}{english}{Professorship for Computer Science}
\addtolanguagecommand{\chair}{ngerman}{Professur f\"ur Informatik}
\newlanguagecommand{\chairsub}
\addtolanguagecommand{\chairsub}{english}{Communication Services, Telecommunication Systems and Computer Networks}
\addtolanguagecommand{\chairsub}{ngerman}{insbesondere Kommunikationsdienste, Telekommunikationsdienste und Rechnernetze}
\newlanguagecommand{\seminar}
\addtolanguagecommand{\seminar}{english}{Seminar on}
\addtolanguagecommand{\seminar}{ngerman}{Ausarbeitung des KTR-Seminars}
\newlanguagecommand{\topic}
\addtolanguagecommand{\topic}{english}{Topic}
\addtolanguagecommand{\topic}{ngerman}{Thema}
\newlanguagecommand{\submitter}
\addtolanguagecommand{\submitter}{english}{Submitted by}
\addtolanguagecommand{\submitter}{ngerman}{Vorgelegt von}
\newlanguagecommand{\lsupervisor}
\addtolanguagecommand{\lsupervisor}{english}{Supervisor}
\addtolanguagecommand{\lsupervisor}{ngerman}{Betreuer}
